\documentclass[12pt]{article}
\usepackage[utf8]{inputenc}
\usepackage{geometry}
\geometry{a4paper, margin=1in}
\usepackage{setspace} % Package pour régler l'interligne
\usepackage{tcolorbox}
\usepackage{xcolor} % Pour utiliser des couleurs
% Pour inclure un logo
\usepackage{graphicx}
\usepackage{tikz} % Pour dessiner
\usepackage{amsmath} % Package pour des maths avancées

\begin{document}

		% Page de garde
		\begin{titlepage}
			
			\begin{tikzpicture}[remember picture, overlay]
				% Dessiner un rectangle autour de la page, plus proche du texte et plus épais
				\draw[thick, color=blue, line width=5mm] 
				(current page.south west) rectangle (current page.north east); % Rectangle de base
				
				\draw[thick, color=blue, line width=5mm] 
				(current page.south west) rectangle ++(-0.5cm, 0.5cm) ++(0.5cm, -0.5cm); % Élargir la bordure
			\end{tikzpicture}
			
			% En haut de la page : logo et texte à gauche, logo et texte à droite
			\noindent
			\begin{minipage}{0.45\textwidth}
				\begin{flushleft} % Alignement à gauche
					
					\begin{center}
						\textbf{\small MINISTERE DE L’ENSEIGNEMENT SUPERIEUR, DE LA RECHERCHE ET DE L’INNOVATION}
					\end{center}
					
					\begin{center}
						\textbf{\small **************}
					\end{center}
					
					\begin{center}
						\textbf{\small DIRECTION GENERALE DE L’ENSEIGNEMENT SUPERIEUR PRIVE}
					\end{center}
					
					\begin{center}
						\textbf{\small **************}
					\end{center}
					
					\begin{center}
						\textbf{\small UNIVERSITE PRIVEE BABA COULIUBBALY}
					\end{center}
					
					\begin{center}
						\begin{center}
							
						\end{center}
						\includegraphics[width=3.5cm]{ubc.png} \\ % Réduire la taille du logo
						\textbf{\small 01BP :3010 BOBO 01}\\
						\textbf{\small TEL: 20-98-52-66}\\
						\textbf{\small Site web: www.isfpburkina.org}    
					\end{center}
					
				\end{flushleft}
			\end{minipage}
			\hfill
			\begin{minipage}{0.45\textwidth}
				\begin{flushright} % Alignement à droite
					
					\begin{center}
						\textbf{\small CENTRE NATIONAL DE LA RECHERCHE SCIENTIFIQUE ET TECHNOLOGIQUE (CNRST)}
					\end{center}
					
					\begin{center}
						\textbf{\small **************}
					\end{center}
					
					\begin{center}
						\textbf{\small INSTITUT DE RECHERCHE EN SCIENCE DE LA SANTE (IRSS)}
					\end{center}
					
					\begin{center}
						\textbf{\small **************}
					\end{center}
					
					\begin{center}
						\textbf{\small DIRECTION REGIONALE DU CENTRE-OUEST (DRCO)}
					\end{center}
					
					\begin{center}
						\textbf{\small **************}
					\end{center}
					
					\begin{center}
						\textbf{\small UNITE DE RECHERCHE CLINIQUE DE NANORO (URCN)}
					\end{center}
					
					\begin{center}
						\includegraphics[width=2cm]{logo-CNRST2.png} \\
						\includegraphics[width=2cm]{irss.png} % Réduire la taille des logos
					\end{center}
				\end{flushright}
			\end{minipage}
			
			% Corps de la page
			\vspace{0.3cm} % Ajouter un peu d'espace entre l'en-tête et le corps
			\begin{center}
				\textcolor{black!70}{\textbf{\LARGE Rapport de fin de cycle}}\\[0.3cm]
				
				\textbf{\normalsize Présenté en vue de l'obtention du Diplôme de Licence professionnelle en Statistique et Informatique}\\[0.5cm]
				
				\textbf{\normalsize Période du stage : Septembre 2024 - Novembre 2024}\\[0.5cm]
				
				\textbf{\normalsize Thème : }\\[0.2cm]
				
				\begin{tcolorbox}[colframe=black, colback=white, sharp corners]
					\textbf{\normalsize Analyse statistique des données : Étude sur ...}
				\end{tcolorbox}
				
				\textbf{\normalsize Présenté par : OUEDRAOGO Aboubacar}\\[1cm]
				
			\end{center}
			
			
			% En bas de page : noms des encadreurs et année académique
			\begin{minipage}{0.45\textwidth} % Bloc gauche pour le maître de stage
				\raggedright
				\textbf{\small Maître de stage :}\\[0.3cm]
				\textbf{\small M. Karim DERRA}\\
				\textbf{\small Statisticien Démographe à l'URCN}
			\end{minipage}
			\hfill
			\begin{minipage}{0.45\textwidth} % Bloc droit pour le directeur de rapport
				\raggedleft
				\textbf{\small Directeur de rapport :}\\[0.3cm]
				\textbf{\small Dr Moussa BARRO}\\
				\textbf{\small Maître de conférences à l’Université Nazi Boni, vacataire à l'ISFP}
			\end{minipage}
			
			\vspace{0.4cm}
			
			\begin{center}
				\textbf{\small}
			\end{center}
			
		\end{titlepage}
	
		% Saut de page après la dédicace
		\newpage
	
	% Numérotation des pages en chiffres romains et petits caractères
	\pagenumbering{roman}
	\renewcommand{\thepage}{\small\roman{page}}
	
	% Dédicace section
	\begin{flushright} % Alignement à droite
		\textbf{\Large Dédicace}
		
		\rule{0.8\linewidth}{0.4pt} % Barre horizontale prenant 50 % de la largeur de la ligne
		\vspace{0.5cm}
		
		% Appliquer l'interligne 2
		\setstretch{2}
		
		À toute ma famille à qui je dois tout ce que je suis aujourd'hui et qui ne cesse de me soutenir en tout temps et tout lieu. \textbf{Famille OUEDRAOGO}.\\
		Je profite de l’occasion pour vous exprimer toute ma reconnaissance.
	\end{flushright}
	
	\vspace{1.5cm}
	
	% Saut de page après la dédicace
	\newpage
	
	\begin{flushright} % Alignement à droite
		% Remerciements section (aligné à gauche par défaut)
		\textbf{\Large Remerciements}

			\rule{0.8\linewidth}{0.4pt} % Barre horizontale prenant 50 % de la largeur de la ligne
		\vspace{0.5cm}
	
	\end{flushright}
	\setstretch{1.5}
	
	La réalisation de ce mémoire a été possible grâce à l'implication de plusieurs personnes à qui
	je voudrais témoigner toute ma gratitude.\\
	
	
	Je voudrais dans un premier temps, remercier mon directeur de memoir, \textbf{Dr Moussa BARRO, Maitre de Conférences en mathématiques appliquées
	à l’Université Nazi Boni}, pour sa patience, sa disponibilité et surtout ses conseils  tout au long du processus de recherche.\\
	
	
	Je voudrais également adresser toute ma gratitude à mon maitre de stage, \textbf{Monsieur Karim DERRA, Statisticien Démographe et manager de la section HDSS de l'URCN}, pour sa confiance, sa disponibilité, et surtout l’autonomie qu’il m’a offert pendant ce stage.\\
	
	
	J'accorde de tout  coeur mes remerciements à l'ensemble du personnel de  l'Unité de Recherche Clinique de Nanoro (URCN) sans distinction de section de travail, de poste ni de grade. \\
	
	
	Je désire aussi remercier les professeurs de l'Institut Superieur des Filières Professionnalisantes, plus particulièrement le corps enseignant de la filière Statistiques-Informatique pour la qualité des enseignements dispensés et leur disponibilité à m'accompagner durant ces trois années de formation.
	
	% Saut de page après la dédicace
	\newpage
	
	\begin{flushright} % Alignement à droite
		% Remerciements section (aligné à gauche par défaut)
		\textbf{\Large Avant-propos}
		
		\rule{0.8\linewidth}{0.4pt} % Barre horizontale prenant 50 % de la largeur de la ligne
		\vspace{0.5cm}
		
	\end{flushright}
	\setstretch{1.5}
	
	\subsection{\textbf{Présentation de la filière}}
	
	
	
	L'Université Privée Baba COULIBALY (UBC) EX ISFP (Institut Supérieur des Filières Professionnalisantes) forme depuis l’année universitaire 2019/2020 des étudiants pour la licence en Statistiques-Informatique. Ils ont pour mission d’assister les cadres supérieurs dans les prises de décision en environnement incertain. Pour accomplir leur mission, ils appliquent aux données recueillis les méthodes d’analyses statistiques et gèrent les bases de donnes des entreprises et des services. Les techniques statistiques appropriées permettent : d’organiser la collecte de l’information, d’analyser, résumer,
	segmenter des vastes ensembles de données, d’écrire, traiter, synthétiser des résultats d’enquête, d’analyser, décomposer, dessaisonnaliser, modéliser des séries chronologiques ; d’estimer
	et tester les effets d’un ensemble de facteurs, de concevoir et planifier un sondage. L’entrée en
	Statistiques et Informatique au niveau de l’Institut Supérieur des Filières Professionnalisantes
	requiert le diplôme du baccalauréat scientifique. Ces professionnels reçoivent des enseignements
	de haut niveau aussi bien sur le plan théorique que pratique. C’est donc dans cette optique
	de lier la théorie à la pratique que des stages de fin de cycle sont instaurés dans les milieux
	professionnels. C’est un moment important dans le parcours du futur statisticien en ce sens
	qu’il lui permet de s’insérer dans le milieu professionnel auquel il est destiné et lui permet aussi
	de se détacher un tant soit peu des notions théoriques qui lui sont enseignés en classe.
	
	
	% Saut de page après la dédicace
	\newpage
	
	\begin{flushright} % Alignement à droite
		% Remerciements section (aligné à gauche par défaut)
		\textbf{\Large Sommaire}
		
		\rule{0.8\linewidth}{0.4pt} % Barre horizontale prenant 50 % de la largeur de la ligne
		\vspace{0.5cm}
		
	\end{flushright}
	
	% Saut de page après la dédicace
	\newpage
	
		\begin{flushright} % Alignement à droite
		% Remerciements section (aligné à gauche par défaut)
		\textbf{\Large Liste des tableaux}
		
		\rule{0.8\linewidth}{0.4pt} % Barre horizontale prenant 50 % de la largeur de la ligne
		\vspace{0.5cm}
		
	\end{flushright}
	
	% Saut de page après la dédicace
	\newpage
	
		\begin{flushright} % Alignement à droite
		% Remerciements section (aligné à gauche par défaut)
		\textbf{\Large Liste des figures}
		
		\rule{0.8\linewidth}{0.4pt} % Barre horizontale prenant 50 % de la largeur de la ligne
		\vspace{0.5cm}
		
	\end{flushright}
	
	% Saut de page après la dédicace
	\newpage
	
	\begin{flushright} % Alignement à droite
		% Remerciements section (aligné à gauche par défaut)
		\textbf{\Large Liste des sigles et abreviations}
		
		\rule{0.8\linewidth}{0.4pt} % Barre horizontale prenant 50 % de la largeur de la ligne
		\vspace{0.5cm}
		
	\end{flushright}
	
	% Saut de page après la dédicace
	\newpage
	
	\begin{flushright} % Alignement à droite
		% Remerciements section (aligné à gauche par défaut)
		\textbf{\Large Résumé}
		
		\rule{0.8\linewidth}{0.4pt} % Barre horizontale prenant 50 % de la largeur de la ligne
		\vspace{0.5cm}
		
	\end{flushright}
	
		% Saut de page après la dédicace
	\newpage
	
		\begin{flushright} % Alignement à droite
		% Remerciements section (aligné à gauche par défaut)
		\textbf{\Large Abstract}
		
		\rule{0.8\linewidth}{0.4pt} % Barre horizontale prenant 50 % de la largeur de la ligne
		\vspace{0.5cm}
		
	\end{flushright}
	
	% Saut de page après la dédicace
	\newpage
	
	\pagenumbering{arabic}
	
	\begin{flushright} % Alignement à droite
		% Remerciements section (aligné à gauche par défaut)
		\textbf{\Large Introduction}
		
		
		\rule{0.8\linewidth}{0.4pt} % Barre horizontale prenant 50 % de la largeur de la ligne
		\vspace{0.5cm}
		
	\end{flushright}
	
	% Saut de page après la dédicace
	\newpage
	
		\begin{flushright} % Alignement à droite
		% Remerciements section (aligné à gauche par défaut)
		\textbf{\Large Approche théorique de l'étude}
		
		
		\rule{0.8\linewidth}{0.4pt} % Barre horizontale prenant 50 % de la largeur de la ligne
		\vspace{0.5cm}
		
	\end{flushright}
	
\end{document}
